\par{La IDT es representada por un arreglo de $idt\_entry$ (esturctura descripta en $Codigo$ $5$) de tamaño 255.}
\\
\begin{lstlisting} [caption={Estructura de idt\_entry},label=idtEntry]
typedef struct str_idt_entry_fld {
    unsigned short offset_0_15;
    unsigned short segsel;
    unsigned short attr;
    unsigned short offset_16_31;
} __attribute__((__packed__, aligned (8))) idt_entry;

\end{lstlisting}

\\
\par{Para inicializar la IDT se hace el llamado a idt\_inicializar, escrita en C. En esta funci\'on lo que hacemos es inicializar todas las entradas de la 0 a la 19, la 32, 33 y 102 utilizando la macro brindada por la c\'atedra. El selector de segmento est\'a definido como el 0x40, correspondiente al codigo de kernel. Los atributos son definidos como 0x8E00 que representan segmento presente (P=1), nivel de privilegio 0 (DPL = 00), 01110 en los bits del 8 al 12 que representan una Interrupt Gate de 32 bits y el resto de los bits en 0 (del 7 al 0).} 

\\
\begin{lstlisting} [caption={Codigo del macro utilizado para inicializar la IDT},label=macro]
#define IDT_ENTRY_SYSTEM(numero)                                                                                        
    idt[numero].offset_0_15 = (unsigned short) ((unsigned int)(&_isr ## numero) & (unsigned int) 0xFFFF);        
    idt[numero].segsel = (unsigned short) 0x0040;                                                                
    idt[numero].attr = (unsigned short) 0x8e00;                                                                  
    idt[numero].offset_16_31 = (unsigned short) ((unsigned int)(&_isr ## numero) >> 16 & (unsigned int) 0xFFFF);
\end{lstlisting}

\\

\par{Las rutinas de atencion estan dadas por una macro que llama a una funci\'on en C que recibe el numero de interrupci\'on e imprime un mensaje dependiendo de que error es, luego queda en un loop infinito. Estas son rutinas dummy que luego fueron modificadas.}
\par{Luego de inicializar la IDT, la cargamos mediante la ejecuci\'on de la instrucci\'on $lidt[IDT\_DESC]$. IDT\_DESC es una estructura que tiene el tamaño de la tabla y la direccion de memoria de la IDT.}

\begin{lstlisting} [caption={Estructura de IDT\_Desc}],label=idtDesc] 
typedef struct str_idt_descriptor {
    unsigned short  idt_length;
    unsigned int    idt_addr;
} __attribute__((__packed__)) idt_descriptor;
\end{lstlisting}