\subsubsection*{Inicializar directorio y tablas de p\'agina para el Kernel}
\par{Antes de activar paginaci\'on debemos inicializar el directorio y tablas de pagina del kernel.
Esto lo haremos con identity mapping de las direcciones 0x00000000 a 0x003FFFFF. El directorio de paginas se posicionar\'a en la direcci\'on 0x27000.}

\par{Para representar al directorio utilizamos un struct llamado $page\_directory\_entry$ y para las tablas utilizamos al struct $page\_table\_entry$. A continuaci\'on las definiciones de estos structs:}

\begin{lstlisting} [caption={Estructura de page\_directory\_entry}],label=idtDesc] 
typedef struct str_page_directory_entry {
    unsigned char   p:1;
    unsigned char   rw:1;
    unsigned char   s:1;
    unsigned char   pwt:1;
    unsigned char   pcd:1;
    unsigned char   a:1;
    unsigned char   i:1;
    unsigned char   ps:1;
    unsigned char   g:1;
    unsigned char   disp:3;
    unsigned int    base:20;
} __attribute__((__packed__, aligned (4))) page_directory_entry;

\end{lstlisting}

\begin{lstlisting} [caption={Estructura de page\_table\_entry}],label=idtDesc] 
typedef struct str_page_table_entry {
    unsigned char   p:1;
    unsigned char   rw:1;
    unsigned char   s:1;
    unsigned char   pwt:1;
    unsigned char   pcd:1;
    unsigned char   a:1;
    unsigned char   d:1;
    unsigned char   pat:1;
    unsigned char   g:1;
    unsigned char   disp:3;
    unsigned int    base:20;
} __attribute__((__packed__, aligned (4))) page_table_entry;


\end{lstlisting}


\newpage
\par{Para inicializar ambas creamos un puntero a $page\_directory\_entry$ y lo seteamos en 0x27000; luego un puntero a $page\_table\_entry$ y lo inicializamos en 0x28000.}

\begin{lstlisting} [caption={Inicializaci\'on de los punteros}],label=idtDesc] 
	page_directory_entry* dir_pagina_kernel = (page_directory_entry*) CR3KERNEL;
	page_table_entry* dir_table_kernel = (page_table_entry*) 0x28000;
\end{lstlisting}
\par{Luego inicializamos el directorio con los siguientes parametros:}

\begin{lstlisting} [caption={Inicializaci\'on del directorio}],label=idtDesc] 
	dir_pagina_kernel->p = 1;
	dir_pagina_kernel->rw = 1;
	dir_pagina_kernel->s = 0;
	dir_pagina_kernel->pwt = 0;
	dir_pagina_kernel->pcd = 0;
	dir_pagina_kernel->a = 0;
	dir_pagina_kernel->i = 0;
	dir_pagina_kernel->ps = 0;
	dir_pagina_kernel->g = 0;
	dir_pagina_kernel->disp = 0;
	dir_pagina_kernel->base = 0x28;
\end{lstlisting}


\\
\par{Como el Kernel ocupar\'a el primer MB, tendremos 1024 tablas (1024 x 4kb) y cada una de ellas la inicializaremos de la siguiente manera:}
\begin{lstlisting} [caption={Inicializaci\'on de las tablas}],label=idtDesc] 
	dir_table_kernel[i].p = 1;
	dir_table_kernel[i].rw = 1;
	dir_table_kernel[i].s = 0;
	dir_table_kernel[i].pwt = 0;
	dir_table_kernel[i].pcd = 0;
	dir_table_kernel[i].a = 0;
	dir_table_kernel[i].d = 0;
	dir_table_kernel[i].pat = 0;
	dir_table_kernel[i].g = 0;
	dir_table_kernel[i].disp = 0;
	dir_table_kernel[i].base = i;
\end{lstlisting}


\\
\subsubsection*{Activar paginaci\'on}
\par{Una vez que tenemos inicializado el mapeo del Kernel podemos pasar a activar paginaci\'on. Para esto debemos activar el bit de PG del registro CR0:}

\begin{lstlisting} [caption={Inicializaci\'on de las tablas}],label=idtDesc] 
    mov eax , cr0
    or eax , 0 x80000000
    mov cr0 , eax
\end{lstlisting}

\subsubsection*{Limpiar buffer de video}
\par{La limpieza del buffer de video y el dibujo del mapa fue hecho utilizando las funciones de print brindadas por la catedra. Estas funciones se encargan de escribir en la memoria del video el string, o n\'umero,  pasado por parametro.}

