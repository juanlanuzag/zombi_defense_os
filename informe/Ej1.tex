\subsubsection*{Inicializaci\'on de la GDT}
\par{Inicializamos la Tabla de Descriptores Globales con entradas para segmentos de codigo de nivel 0 y 3, otras para segmentos de datos de nivel 0 y 3 y una para un segmento que describe el area de la pantalla de video. Los segmentos de datos y c\'odigos estan organizados de tal forma que se superpongan direccionando los primeros 623MB de memoria (Sistema FLAT), utilizando bloques de 4K al setear el bit de granularidad en 1.}
\par{A continuaci\'on explicamos c\'omo completamos cada uno de los campo de las entradas de la gdt:}
\begin{itemize}
    \item \textbf{Base y limite} Como mencionamos anteriormente, los segmentos de codigo y datos est\'an superpuestos. Comienzan en la direccion base 0x00000000. El valor del l\'imite 0x26348 corresponde la cantidad de bloques de 4k, este valor nos da los 623 MB que queremos.
    \item \textbf{Tipo:}  El tipo para los segmentos de codigo es 0x08 (executable), mientras que para los de datos y video es 0x02 (readable, writable).
    \item \textbf{Sistema:}  El bit de system esta seteado en 1 salvo en los segmentos correspondientes a las tss de las tareas, donde esta activo bajo en 0 pues son potestad exclusiva del sistema operativo (veremos las tss mas adelante).
    \item \textbf{DPL:}   Los segmentos de datos y codigo de nivel kernel tienen DPL en 0x00, al igual que los segmentos de sistema y el de video, mientras que los de codigo y datos nivel usuario tienen DPL en 0x03.    
    \item \textbf{Granularidad:} El bit de G esta activo s\'olo en los segmentos de datos y c\'odigo ya que es necesario para tener segmentos del tamaño indicado en la consigna.
    \item \textbf{P, L, D/B, AVL::}  Seteados en 1, 0, 1 y 0 respectivamente para todas las entradas.
\end{itemize}

\par{Para representar a la GDT tenemos un arreglo de gdt\_entry, donde gdt\_entry es la siguiente estructura:}

\begin{lstlisting} [caption={Estructura de gdt\_entry},label=idtEntry]
typedef struct str_gdt_entry {
    unsigned short  limit_0_15;
    unsigned short  base_0_15;
    unsigned char   base_23_16;
    unsigned char   type:4;
    unsigned char   s:1;
    unsigned char   dpl:2;
    unsigned char   p:1;
    unsigned char   limit_16_19:4;
    unsigned char   avl:1;
    unsigned char   l:1;
    unsigned char   db:1;
    unsigned char   g:1;
    unsigned char   base_31_24;
} __attribute__((__packed__, aligned (8))) gdt_entry;
\end{lstlisting}

\\
\subsubsection*{Pasaje a modo protegido}

\par{Para pasar a modo protegido completamos y cargamos la GDT con la instrucci\'on lgdt, que toma el descriptor de la GDT dado por el siguiente struct:}

\begin{lstlisting} [caption={Estructura de gdt\_descriptor},label=idtEntry]
typedef struct str_gdt_descriptor {
    unsigned short  gdt_length;
    unsigned int    gdt_addr;
} __attribute__((__packed__)) gdt_descriptor;

\end{lstlisting}

\\
\par{Luego habilitamos A20 para tener acceso a direcciones superiores a $2^{20}$, seteamos el bit PE de CR0 y saltamos a 0x40:mp. 0x40 es el selector de segmento de c\'odigo de nivel 0.}

\begin{lstlisting}[caption={Bloque para saltar a modo protegido},label=idtEntry]
    ; Habilitar A20
    call habilitar_A20
    
    ; Cargar la GDT
    lgdt [GDT_DESC]
    ; Setear el bit PE del registro CR0
    MOV eax,cr0 
    OR eax,1 
    MOV cr0,eax
    ; Saltar a modo protegido
    jmp 0x40:mp
\end{lstlisting}

\\
\par{Luego procedemos a setear los selectores de segmentos de datos de nivel 0 (indice 9 de la gdt seguido de 3 0 correspondientes a TI y RPL). Luego el selector de segmentos fs (indice 12). Por ultimo establecemos la pila en 0x27000.}
\begin{lstlisting}[caption={Bloque para setear los selectores de segmento},label=idtEntry]
    mp:
        xor eax, eax
        mov ax, 0x48 ;1001000b
        mov ds, ax
        mov es, ax
        mov gs, ax
        mov ss, ax  
        mov ax, 1100000b
        mov fs, ax

    ; Establecer la base de la pila
        mov esp, 0x27000
\end{lstlisting}



