\subsubsection*{Asociamos las rutinas de interrupcion a la de reloj, teclado y syscall}
\par{Para inicializar las interrupciones de teclado y reloj utilizamos la misma macro nombrada en el Ejercicio2, llamada IDT\_ENTRY\_SYSTEM. Para inicializar la entrada de la syscall creamos una macro parecida a la IDT\_ENTRY\_SYSTEM pero con distintos atributos. A esta la llamaremos IDT\_ENTRY\_USER y la describimos de la siguiente manera:}

\begin{lstlisting} [caption={Macro IDT\_ENTRY\_USER}],label=idtDesc] 
#define IDT_ENTRY_USER(numero)                                         
    idt[numero].offset_0_15 = (unsigned short) ((unsigned int)(&_isr ## numero) & (unsigned int) 0xFFFF); 
    idt[numero].segsel = (unsigned short) 0x0040;
    idt[numero].attr = (unsigned short) 0xEE00;
    idt[numero].offset_16_31 = (unsigned short) ((unsigned int)(&_isr ## numero) >> 16 & (unsigned int) 0xFFFF);
\end{lstlisting}


\\
\subsubsection*{Rutina atenci\'on de clock}
\par{Creamos la rutina de atenci\'on de clock, en esta llamamos a pr\'oximo reloj que se encarg\'a de avanzar el dibujo del clock y luego avisamos al PIC que la interrupci\'on fue atendida mediante call fin\_intr\_pic1. Luego ser\'a modificada para el uso del scheduler.}
\begin{lstlisting} [caption={Rutina atenci\'on clock}],label=idtDesc] 
_isr32:
    pushad
    call proximo_reloj
    call fin_intr_pic1
    popad
    iret

\end{lstlisting}

\subsubsection*{Rutina de atenci\'on de teclado}
\par{Creamos la rutina de atenci\'on de teclado. Esta rutina toma el valor de la tecla presionada y compara este valor con los de las teclas correspondientes al juego. Luego hace un jump a el c\'odigo de esa tecla en particular. A continuaci\'on un ejemplo de la tecla $a$:}

\begin{lstlisting} [caption={Rutina atenci\'on teclado}],label=idtDesc] 
_isr33:
    pushad
    xor eax, eax
    in al, 0x60
    
    cmp al, 0x1e
    je .a

.a: ; A hacia izq
    mov ebx, -1
    push ebx
    mov ebx, 0
    push ebx
    call game_change_zombie
    add esp, 8
    jmp .fin


    .fin:
        call fin_intr_pic1
        popad
        iret
\end{lstlisting}

\\
\par{El c\'odigo real es m\'as largo, contiene comparaciones y etiquetas para todas las teclas que son parte del juego.}

\subsubsection*{Rutina atenci\'on syscall}
\par{La siguiente es la rutina de atenci\'on de la interrupci\'on 0x66, este es el c\'odigo ya modificado para atender la syscall. Luego de mover al zombie hacemos un salto a la tarea idle.}
\begin{lstlisting} [caption={Rutina atenci\'on de  syscall}],label=idtDesc] 
_isr102:
    pushad
    push eax
    call game_move_current_zombi
    pop eax
    ;jump a la tarea idle
    jmp 0x70:0xFAFAFA    
    popad
    iret

\end{lstlisting}
